\documentclass{article}
\author{Gabriel Haruo Hanai Takeuchi - NUSP: 13671636}
\title{MAC0105 - Exercícios}
\date{}

\usepackage[utf8]{inputenc}
\usepackage[a4paper, margin=3cm]{geometry}
\usepackage[skip=5pt]{parskip}
\usepackage{amsmath, amssymb, amsthm}
\usepackage{graphicx}
\graphicspath{{./images}}
\usepackage{wrapfig}
\usepackage{braket}

\newcommand{\Z}{\mathbb{Z}}
\newcommand{\N}{\mathbb{N}}

\begin{document}

\maketitle

\section*{Exercício 2}
Prove que dado um inteiro $n$ ímpar, vale que
\[
1+2+\dots+n = n(n+1)/2.
\]
\textbf{Obs}.: Para quem sabe o que é \textit{indução}, não é para
usar neste exercício.

\hrulefill

\begin{proof}
Some o primeiro e último elemento da sequência: temos $1 + n$.
Agora, some o segundo e penúltimo: temos $2 + (n-1) = 1 + n$.
Facilmente, é possível fazer uma indução, mas a pura intuição basta para o problema.
Note que estamos extraindo $\frac{n-1}{2}$ \textbf{pares} da sequência, mas temos um $n$ ímpar, enão resta o número do meio $\frac{n+1}{2}$.
Logo,
\begin{align*}
&1 + 2 + 3 + \dots + (n-1) + n \\
&= \dfrac{n-1}{2} \cdot (n+1) + \dfrac{n+1}{2} \\
&= \dfrac{(n+1)(n-1 + 1)}{2} \\
&= \dfrac{(n+1)n}{2}
\end{align*}
\end{proof}

\section*{Exercício 4}

Dizemos que $A$ e $B$ são conjuntos \emph{comparáveis} se $A\subseteq B$ ou $B\subseteq A$.
Escreva com suas palavras uma demonstração do seguinte resultado:

\textit{Seja $X$ um conjunto finito com $n$ elementos.
Suponha que $A_1,A_2,\dots,A_n,A_{n+1},A_{n+2}$ são subconjuntos de $X$, todos
distintos entre si.
Então existem $i$ e $j$ com $1\leq i\leq n+2$ e $1\leq j\leq n+2$ tais que $A_i$ e $A_j$ \textbf{não} são comparáveis.}

\begin{proof}

\end{proof}

\section*{Exercício 30}
Prove que para todo inteiro positivo $n$ vale que $n^3 \leq 3^n$.
\begin{proof}
Vamos provar por indução forte.

\textbf{Base: } Para $n=1$, diretamente $1^3 = 1 \leq 3 = 3^1$.
Para $n=2$, diretamente $2^3 = 8 \leq 9 = 3^2$

\textbf{Passo: } Fixe $n \geq 3$ e suponha a H.I. para $n-1$.

\begin{center}
(H.I.): $(n-1)^3 < 3^{n-1}$.
\end{center}

\begin{align*}
    n^3 = \Bigl( \dfrac{n (n-1)}{(n-1)} \Bigr)^3 = n^3 \dfrac{n-1}{n-1}
\end{align*}

\end{proof}

\section*{Exercício 38}
\begin{proof}
    
\end{proof}

\end{document}