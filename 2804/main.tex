\documentclass{article}
\author{Gabriel Haruo Hanai Takeuchi - NUSP: 13671636}
\title{MAC0105 - Exercícios}
\date{}

\usepackage[utf8]{inputenc}
\usepackage[a4paper, margin=3cm]{geometry}
\usepackage[skip=5pt]{parskip}
\usepackage{amsmath, amssymb, amsthm}

\begin{document}
\maketitle

\section*{Exercício 11}
Escreva a tabela verdade das seguintes proposições:

(a) $p \implies (q \lor r)$

\textit{Resposta:}
\begin{displaymath}
\begin{array}{|c c c|c|}
% |c c|c| means that there are three columns in the table and
% a vertical bar ’|’ will be printed on the left and right borders,
% and between the second and the third columns.
% The letter ’c’ means the value will be centered within the column,
% letter ’l’, left-aligned, and ’r’, right-aligned.
p & q  & r & p \implies (q \lor r)\\ % Use & to separate the columns
\hline % Put a horizontal line between the table header and the rest.
T & T & T & T\\
T & T & F & T\\
T & F & T & T\\
F & T & T & T\\
T & F & F & F\\
F & T & F & T\\
F & F & T & T\\
F & F & F & T\\
\end{array}
\end{displaymath}

(b) $\neg q \implies (\neg q \lor r)$

\textit{Resposta:}
\begin{displaymath}
\begin{array}{|c c c c|c|}
p & q  & r & \neg q \lor r & \neg q \implies (\neg q \lor r)\\
\hline
T & T & T & T & T\\
T & T & F & F & T\\
T & F & T & T & T\\
F & T & T & T & T\\
T & F & F & T & T\\
F & T & F & F & T\\
F & F & T & T & T\\
F & F & F & T & T\\
\end{array}
\end{displaymath}
A sentença $\neg q \implies (\neg q \lor r)$ é uma tautologia.

(c) $(p \land (p \implies q)) \implies q$

\textit{Resposta:}
\begin{displaymath}
\begin{array}{|c c c c|c|}
p & q & p \implies q & p \land (p \implies q) & (p \land (p \implies q)) \implies q\\
\hline
T & T & T & T & T\\
T & F & F & F & T\\
F & T & T & F & T\\
F & F & T & F & T\\
\end{array}
\end{displaymath}
A sentença $(p \land (p \implies q)) \implies q$ é uma tautologia.

(d) $p \implies q$

\textit{Resposta:}
\begin{displaymath}
\begin{array}{|c c|c|}
p & q & p \implies q\\
\hline
T & T & T\\
T & F & F\\
F & T & T\\
F & F & T\\
\end{array}
\end{displaymath}

(e) $\neg (p \implies q)$

\textit{Resposta:}
\begin{displaymath}
\begin{array}{|c c|c|}
p & q & \neg (p \implies q)\\
\hline
T & T & F\\
T & F & T\\
F & T & F\\
F & F & F\\
\end{array}
\end{displaymath}

(f) $\neg p \land q$

\textit{Resposta:}
\begin{displaymath}
\begin{array}{|c c|c|}
p & q & \neg p \land q\\
\hline
T & T & F\\
T & F & F\\
F & T & T\\ F & F & F\\ \end{array} \end{displaymath}

\section*{Exercício 12}

Prove os seguintes itens de forma direta ou utilizando a contrapositiva.
Pense bem qual a melhor forma de provar cada item.

(a) Se $x$ é ímpar, então $x^2$ é ímpar.
\begin{proof}
Vamos provar diretamente.

Suponha que $x$ seja ímpar.
Logo, existe $k \in \mathbb{N}$ tal que $x = 2k + 1$.
Multiplicando $x$ por $x$, temos
\begin{align*}
    x^2 &= (2k + 1)^2\\
    &= (4k^2 + 4k) + 1\\
    &= 2(2k^2 + 2k) + 1 .
\end{align*}
Observe que $2k^2 + 2k \in \mathbb{N}$.
Note que $x^2$ é da forma $2m + 1$, sendo $m = 2k^2 + 2k \in \mathbb{N}$.
Logo, $x^2$ é da forma de um número ímpar, como queríamos.

\end{proof}

(b) Suponha que $x$ e $y$ são números reais. Se $y^3 + yx^2 \leq x^3 + xy^2$, então $y \leq x$.
\begin{proof}
Vamos provar por contrapositiva, ou seja, $y > x \implies y^3 + yx^2 > x^3 + xy^2$.

Suponha que $y > x$.
Logo,
\begin{align*}
    y > x &\implies y(y^2 + x^2) > x(x^2 + y^2) , \quad \mbox{isto pois }(x^2 + y^2) > 0\\
    &\implies y^3 + yx^2 > x^3 + xy^2
\end{align*}
Como queríamos.
\end{proof}

(c) Sejam $x$, $y$ e $z$ números inteiros. Se $x \mid y$ e $y \mid z$, então $x \mid z$ ($a \mid b$ significa que $a$ divide $b$).

\begin{proof}
Vamos provar diretamente.

Suponha que $x \mid y$.
Logo, $\exists a \in \mathbb{Z}$ tal que $y = a \cdot x$.
Suponha que $y \mid z$, ou seja, $a \cdot x \mid z$.
Logo, $\exists b \in \mathbb{Z}$ tal que $z = b \cdot a \cdot x$.
Note que $z = (b \cdot a) \cdot x$, onde $ab \in \mathbb{Z}$.
Portanto, $x \mid z$.

\end{proof}

\section*{Exercício 15}

Seja $p$ uma proposição como abaixo:
\[ p = \forall x,y \in \mathbb{N}, ((x<y) \implies (\exists z \in \mathbb{N}, x<z<y)) \]
Faça o que é pedido nos itens abaixo:

(a) Escreva a negação $\neg p$ de $p$:

\textit{Resposta: }
\begin{align*}
&\neg(\forall x,y \in \mathbb{N}, ((x<y) \implies (\exists z \in \mathbb{N}, x<z<y)))\\
&\equiv \neg(\forall x,y \in \mathbb{N}, (\neg(x<y) \lor (\exists z \in \mathbb{N}, x<z<y))) \qquad \mbox{note que }(p \implies q \equiv \neg p \lor q) \\
&\equiv \exists x,y \in \mathbb{N}, ((x<y) \land \neg (\exists z \in \mathbb{N}, x<z<y)) \\
&\equiv \exists x,y \in \mathbb{N}, ((x<y) \land \neg (\exists z \in \mathbb{N}, z>x \land z<y)) \\
&\equiv \exists x,y \in \mathbb{N}, ((x<y) \land (\forall z \in \mathbb{N}, z \leq x \lor z \geq y))
\end{align*}

(b) Escreva em língua portuguesa, com palavras, o significado de $p$ e de $\neg p$:

\textit{Resposta:}

Em português, $p$ é:

Para quaisquer $x,y$ naturais, se $x$ é menor que $y$, então existe $z$ natural tal que $z$ está entre $x$ e $y$.

Em português, $\neg p$ é:

Existe pelo menos um par $x,y$ de naturais tais que $x$ é menor que $y$ e todo natural é menor-igual a $x$ ou maior-igual a $y$.

(c) $p$ é verdadeira? Justifique.

\textit{Resposta:}

A proposição $p$ é falsa. Vamos mostrar um contraexemplo:

Suponha $x=1, y=2$.
Não existe natural entre $x$ e $y$ que seja diferente de $x$ e $y$.

(d) $\neg p$ é verdadeira? Justifique.

\textit{Resposta:}

A proposição $\neg p$ é verdadeira.
Basta mostrar um exemplo de $x,y$ que satisfaça as condições impostas.

O exemplo sempre ocorre se $y=x+1$.
Portanto, suponha $x,y = x+1$.
Perceba que qualquer número natural está no intervalo $[0,x] \cup [y,+\infty] = [0,x] \cup [x+1,+\infty]$.
Logo, existem $x,y$ tais que $(x < y)$ e $(\forall z \in \mathbb{N}, z \leq x \lor z \geq y)$.

Portanto, $\neg p$ é verdadeiro.

\end{document}