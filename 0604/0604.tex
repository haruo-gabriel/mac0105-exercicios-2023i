
\documentclass{article}
\author{Gabriel Haruo Hanai Takeuchi - NUSP: 13671636}
\title{MAC0105 - Exercícios para 17/05}
\date{}

\usepackage[utf8]{inputenc}
\usepackage[a4paper, margin=2cm]{geometry}
\usepackage[skip=5pt]{parskip}
\usepackage{amsmath, amssymb, amsthm}
\usepackage{graphicx}


\newcommand{\set}[1]{\{#1\}}
\newcommand{\base}{\textbf{Base: }}
\newcommand{\passo}{\textbf{Passo: }}
\newcommand{\Mod}[1]{\ (\mathrm{mod}\ #1)}

\begin{document}
\maketitle

\section*{Exercício 49}
Prove os seguintes itens.

(a) Seja $n$ um inteiro positivo. Se a diferença entre as somas dos dígitos de $n$ em posição par e a soma dos dígitos de $n$ em posição ímpar é múltiplo de $11$, então $n$ é múltiplo de $11$;

\begin{proof}
Vamos convencionar uma notação para os algarismos de $n$ (ou qualquer inteiro): seja $n = [ a_{n}, a_{n-1}, \dots, a_2, a_1 ]$ tal que $a_1$ é o algarismo da unidade, $a_2$ é a dezena, e assim por diante.

Suponha um $n$ tal que 
\begin{align*}
( a_1 + a_3 + \dots + a_{n} ) -( a_2 + a_4 + \dots + a_{n-1}) \equiv 0 \Mod{11} \qquad (I)
\end{align*}
e queremos chegar em
\begin{align*}
  10^{n-1} a_n + 10^{n-2} a{n-1} + \dots + 10^1 a_2 + 10^0 a_1 \equiv 0 \Mod{11}.
\end{align*}

Multiplicando (I) por $10^{n-1}$, o resultado ainda é múltiplo de 11:
\begin{align*}
&10^{n-1} [ ( a_1 + a_3 + \dots + a_{n} ) -( a_2 + a_4 + \dots + a_{n-1}) ] \equiv 0 \Mod{11} \\
\implies &(10^0)(10^{n-1}a_{n}) + (-10^1)(10^{n-2}a_{n-1}) + \dots + (-10^{n-2})(10^{1}a_2) + (10^{n-1})(10^{0}a_1) \equiv 0 \Mod{11} \\
\end{align*}
Observe que $10 \Mod{11} = -1$. Logo, se $x$ é ímpar, então $10^x \Mod{11} = -1$, e se $x$ é par, então $10^x \Mod{11} = 1$.
Com isso, podemos reduzir as potências de 10 para 1 ou -1.
\begin{align*}
  &(10^0)(10^{n-1}a_{n}) + (-10^1)(10^{n-2}a_{n-1}) + \dots + (-10^{n-2})(10^{1}a_2) + (10^{n-1})(10^{0}a_1) \equiv 0 \Mod{11} \\
  \implies &10^{n-1}a_{n} + 10^{n-2}a_{n-1} + \dots + 10^{1}a_2 + 10^{0}a_1 \equiv 0 \Mod{11}
\end{align*}
\end{proof}

(b) $11$ é o único número primo palíndromo com uma quantidade par de dígitos.

\begin{proof}
Inicialmente, considere um número palíndromo $p$ com $2n$ dígitos.
Usando a notação exibida no exercício 51 para representar os algarismos de $p$, temos
\[ p = [a_1 , a_2 , \dots , a_{n} , a_{n+1} , a_{n+2} , \dots , a_{2n} ] \]
sendo $a_{2n}$ o algarismo das unidades, $a_{2n-1}$ o algarismo das dezenas, etc.

Note que, por $p$ ser palíndromo, então 
\[ p = [ a_1, a_2, \dots, a_n, a_n, a_{n-1}, \dots, a_1 ] \]

Note que a diferença entre as somas dos dígitos de $p$ em posição par e a soma dos dígitos de $n$ em posição ímpar é igual a 0.
Como 0 é múltiplo de 11, então $p$ é múltiplo de 11.
Logo, o único $p$ (palíndromo com quantidade par de dígitos) que é primo é o próprio 11.

\end{proof}

\section*{Exercício 51}

Prove ou dê um contraexemplo para os itens abaixo, em que $n\in\mathbb{N}$.

(a) $3^{4n+1} + 2^{8n+3}\equiv 1\pmod 5$;
\begin{proof}
Lembre-se que, para um número $x$ ser divisível por 5, então o algarismo da unidade de $x$ deve ser 0 ou 5.
Para $3^{4n+1} + 2^{8n+3} \equiv 1\pmod 5$, então basta que o algarismo da unidade de $3^{4n+1} + 2^{8n+3}$ seja $0+1 = 1$ ou $5+1=6$.

Agora, vamos provar que:
\begin{itemize}
  \item O algarismo da unidade de $3^{4n+1}$ é sempre 3;
  \item O algarismo da unidade de $2^{8n+3}$ é sempre 8;
\end{itemize}
Dessa forma, o algarismo da soma sempre será 1, como queremos.

\begin{proof} Vamos provar por indução em $n$ que o algarismo da unidade de $3^{4n+1}$ é sempre 3:
\base Para $n=0$, temos que $3^{4n+1} = 3$.

\passo Fixe $n=1$ e suponha para $n-1$.
\end{proof}

\end{proof}

(b) $2^{3n}+6\equiv\pmod 7$;

(c) Se $n\geq 4$, então para todo $x\in\mathbb{Z}$ vale que se $x^2\equiv 4\pmod n$, então $x\equiv 2\pmod n$.

(d) Existe $m\in\mathbb{Z}$ positivo tal que  $\forall x\in\mathbb{Z}$, se $x^2\equiv 4\pmod m$, então $x\equiv 2\pmod m$.

\section*{Exercício 52}

Prove que as relações $R$ definidas nos itens abaixo são relações de equivalência e \textbf{descreva as classes de equivalência} de $R$ em cada item.

(a) $R=\{(x,y)\in \mathbb{Z}\times \mathbb{Z}\colon x^2\equiv y^2\pmod 4\}$;

(b) $R=\{(x,y)\in \mathbb{R}\times\mathbb{R}\colon x-y\in\mathbb{Z}\}$.

(c) $R=\{(x,y)\in \mathbb{Z}\times \mathbb{Z}\colon 3x-5y \equiv 0\pmod 2\}$;

(d) $R=\{(x,y)\in \mathbb{Z}\times \mathbb{Z}\colon x^2+y^2\equiv 0\pmod 2\}$.

\end{document}