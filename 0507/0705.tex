\documentclass{article}
\author{Gabriel Haruo Hanai Takeuchi - NUSP: 13671636}
\title{MAC0105 - Exercícios para 07/05}
\date{}

\usepackage[utf8]{inputenc}
\usepackage[a4paper, margin=2cm]{geometry}
\usepackage[skip=5pt]{parskip}
\usepackage{amsmath, amssymb, amsthm}

\newcommand{\set}[1]{\{#1\}}

\begin{document}
\maketitle

\section*{Exercício 18}
Dadas casas de pombo $h_1,\dots, h_t$ e inteiros $n_1,\dots, n_t$, se $1 + \sum^{t}_{i=1}(n_i - 1)$ pombos são distribuídos entre as $t$ casas de pombo, então existe $i$ tal que $h_i$ contém pelo menos $n_i$ pombos.
\begin{proof}
Vamos iniciar avaliando a quantidade máxima de pombos em uma casa. Se distribuídos igualmente (na medida do possível, pois estamos lidando com inteiros) em todas as casas, a quantidade máxima de pombos em uma casa é
\begin{align*}
    \Big\lceil \dfrac{1 + \sum^{t}_{i=1}(n_i - 1)}{t} \Big\rceil = \Big\lceil \dfrac{1 - t + \sum^{t}_{i=1}n_i}{t} \Big\rceil = \Big\lceil \dfrac{1}{t} - 1 + \dfrac{1}{t}\sum^{t}_{i=1}n_i \Big\rceil .
\end{align*}
Vamos separar em casos, de acordo com o critério dos $n_i$.

\textbf{Caso I:}
Se $n_1 = \dots = n_i$, então
\begin{align*}
    \Big\lceil \dfrac{1}{t} - 1 + \dfrac{1}{t}\sum^{t}_{i=1}n_i \Big\rceil = \Big\lceil \dfrac{1}{t} - 1 + n_1 \Big\rceil = n_1 = \dots = n_i .
\end{align*}
Note que $0 < \frac{1}{t} \leq 1$, logo $ -1 < \frac{1}{t} - 1 \leq 0$, e portanto o teto acima é exatamente igual a $n_1 = \dots = n_i$.

Observe que $\frac{1}{t}\sum^{t}_{i=1}n_i$ é a média aritmética entre os $n_i$.
Como supomos que todos são iguais, então esse somatório é igual a qualquer $n_i$ (particularmente, é igual a $n_1$).

O resultado que tivemos é que o número máximo de pombos em cada casa é $n_1$ (mais especificamente, todas as casas têm exatamente $n_1$ pombos).
Logo, claramente existe uma casa com pelo menos $n_i$ pombos, como queríamos.

\textbf{Caso II:} Se os $n_i$ não são todos iguais, então existe um menor $n_i$.
Seja $n_m$ o menor dos $n_i$'s. então
\begin{align*}
    \Big\lceil \dfrac{1}{t} - 1 + \dfrac{1}{t}\sum^{t}_{i=1}n_i \Big\rceil \geq \Big\lceil \overbrace{\dfrac{1}{t} - 1}^{> -1 \mbox{ e } \leq 0} + n_m \Big\rceil = n_m .
\end{align*}

Logo, existe um $i$ desejado tal que $h_i$ contém no mínimo $n_i$ pombos. Nesse caso, $i=m$, o índice do menor $n_i$.

Cobertos todos os casos, está demonstrado o resultado.
\end{proof}

\section*{Exercício 22}
Prove que para quaisquer $n + 1$ números escolhidos de $\{1, 2,\dots, 2n - 1\}$, existem 2 desses números cuja soma é igual a $2n$.

% tentativa 3
\begin{proof}
Note: os pares que somados resultam em $2n$ são os elementos de $A$ tal que
\[ A = \set{ \set{1,2n-1}, \set{2,2n-2}, \set{3,2n-3}, \dots, \set{n-1,n+1} }\]
Observe que há $n-1$ elementos em $A$. Note que único elemento de \set{1, 2,\dots, 2n - 1} que não possui par em $A$ é $n$.

Por exaustão, vamos escolher $n+1$ elementos de $A$ com o objetivo de não termos 2 números que somam $2n$, ou seja, que não formem um par não-ordenado de $A$.
Perceba que pelo princípio da casa dos pombos, o máximo de elementos que podemos pegar e não termos um par de $A$ é $n$.
O $n$-ésimo primeiro termo obrigatoriamente formará um par que some $2n$.

Logo, sempre existirão 2 números cuja soma resulta em $2n$, como queríamos.
\end{proof}


\section*{Exercício 24}
Prove que a soma dos termos de uma progressão geométrica com $n$ termos, termo inicial a e razão $q$ é dada por
\[ \dfrac{a(q^n - 1)}{q - 1} \]
\begin{proof}
Note que a a soma dos termos de uma progressão geométrica com $n$ termos é
\[ S_n = \sum_{i=0}^{n-1} a q^i = a + aq + aq^2 + \dots + aq^{n-1} . \qquad (i)\]
Multiplicando os dois lados por $q$, temos
\[ q S_n = aq + aq^2 + \dots + aq^n . \qquad (ii)\]
Subtraindo (i) de (ii), temos 
\begin{gather*}
S_n - q S_n  = a + aq + \dots + aq^{n-1} - (aq + aq^2 + \dots + aq^n) \\
S_n (1 - q) = a - aq^n \\
S_n  = \dfrac{a - aq^n}{(1 - q)} = \dfrac{a(1 - q^n)}{(1 - q)}
\end{gather*}
Multiplicando por $\frac{-1}{-1}$, temos finalmente que
\[ S_n = \dfrac{a(q^n - 1)}{(q-1)}, \]
como queríamos.
\end{proof}

\section*{Exercício 25}
O número harmônico de ordem $n$, denotado por $H_n$, é definido como $H_n = \sum_{i=1}^{n}(1/i)$.
Prove que $H_{2^n} \geq 1 + n/2$ para todo inteiro $n \geq 0$.

\begin{proof}
Vamos provar por indução.

\textbf{Base: } Para $n=0$,
\[ H_{2^n} = H_1 = \sum_ {i=1}^{1} \dfrac{1}{i} = 1 \geq \dfrac{1}{2} . \]

\textbf{Passo: } Suponha que $H_{2^n} \geq 1 + n/2$. Vamos provar que $H_{2^{(n+1)}} \geq 1 + (n+1)/2$.
\begin{align*}
&H_{2^n} = 1 + \dfrac{1}{2} + \dfrac{1}{3} + \dots + \dfrac{1}{2^n} \geq 1 + \dfrac{n}{2} \\
&\implies 1 + \dfrac{1}{2} + \dfrac{1}{3} + \dots + \dfrac{1}{2^n} + \Big(\dfrac{1}{2}\Big) \geq 1 + \dfrac{n}{2} + \Big(\dfrac{1}{2}\Big) \\
&\implies 1 + \dfrac{1}{2} + \dfrac{1}{3} + \dots + \dfrac{1}{2^n} + \Big(\dfrac{1}{2}\Big) \geq 1 + \dfrac{n+1}{2} \\
\end{align*}
Note que $\frac{1}{2}$ é sempre maior que $\frac{1}{2^{(n+1)}}$ para todo $n \geq 0$.
Logo, temos o resultado
\begin{align*}
    &1 + \dfrac{1}{2} + \dfrac{1}{3} + \dots + \dfrac{1}{2^n} + \Big(\dfrac{1}{2}\Big) \geq 1 + \dfrac{1}{2} + \dfrac{1}{3} + \dots + \dfrac{1}{2^n} + \Big(\dfrac{1}{2^{(n+1)}}\Big) \geq 1 + \dfrac{n+1}{2} \\
    &\implies H_{2^{(n+1)}} \geq  1 + \dfrac{n+1}{2} ,
\end{align*}
como queríamos.
\end{proof}

\section*{Exercício 26}
Mostre que para todo inteiro positivo $n$ vale que \[ \sum_{i=1}^{n}i^2 = \dfrac{n}{6} (n+1) (2n+1) \]
\begin{proof}
Vamos provar por indução.

Antes de iniciar a prova em si, note que $\frac{n}{6}(n+1)(2n+1) = \dfrac{2n^3 + 3n^2 + n}{6}$.
Vamos usar esse resultado para facilitar o passo indutivo.

\textbf{Base: }Suponha n = 1.

Diretamente, $\sum_{i=1}^{1}i^2 = 1 = \dfrac{1}{6} (1+1) (2*1+1)$.

\textbf{Passo: }Fixe $n \geq 2$ e suponha que vale para $n-1$.

Por hipótese,
\begin{align*}
    \sum_{i=1}^{n-1}i^2 = \dfrac{(n-1)((n-1)+1) (2(n-1)+1)}{6}  = \dfrac{(n-1)n(2n-1)}{6} \\
\end{align*}
Logo,
\begin{align*}
    \implies \sum_{i=1}^{n-1}i^2 + n^2 = \dfrac{(n-1)n(2n-1)}{6} + n^2 = \dfrac{2n^3 - 3n^2 + n}{6} + \dfrac{6n^2}{6} = \dfrac{2n^3 + 3n^2 + n}{6} , \\
\end{align*}
como queríamos (de acordo com a observação pré-prova).

\end{proof}

\end{document}