\documentclass{article}
\author{Gabriel Haruo Hanai Takeuchi - NUSP: 13671636}
\title{MAC0105 - Exercícios para 17/05}
\date{}

\usepackage[utf8]{inputenc}
\usepackage[a4paper, margin=2cm]{geometry}
\usepackage[skip=5pt]{parskip}
\usepackage{amsmath, amssymb, amsthm}
\usepackage{graphicx}


\newcommand{\set}[1]{\{#1\}}
\newcommand{\base}{\textbf{Base: }}
\newcommand{\passo}{\textbf{Passo: }}

\begin{document}
\maketitle

\section*{Exercício 41}
Prove que para todo inteiro positivo $n$, vale que $13^n$ pode ser escrito como a soma de dois quadrados.

\begin{proof}
 
\end{proof}

\section*{Exercício 42}
Uma máquina automática é preenchida com uma quantidade ímpar de exercícios de FUMAC
e uma quantidade ímpar de exercícios de Cálculo I. Sempre que colocamos uma ficha na
máquina, ela libera 2 exercícios de uma vez. A máquina só é abastecida após ficar vazia.
Prove que, antes de ficar vazia, a máquina irá liberar pelo menos um par que é composto por
um exercício de FUMAC e um exercício de Cálculo I.

\begin{proof}
    
\end{proof}

\section*{Exercício 47}
Prove que para todo inteiro $n \geq 2$, se $x_1,\dots, x_n$ são números reais, então
\[
\sqrt{x_i^2 + \dots + x_n^2} \leq |x_1| + \dots + |x_n| \; .
\]
Obs.: Pode ser útil considerar um número positivo $z$ tal que $z^2 = \sum_{i=1}^{n-1} x_i^2$ .

\begin{proof}
    
\end{proof}

\end{document}