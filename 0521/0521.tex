\documentclass{article}
\author{Gabriel Haruo Hanai Takeuchi - NUSP: 13671636}
\title{MAC0105 - Exercícios para 17/05}
\date{}

\usepackage[utf8]{inputenc}
\usepackage[a4paper, margin=2cm]{geometry}
\usepackage[skip=5pt]{parskip}
\usepackage{amsmath, amssymb, amsthm}
\usepackage{graphicx}


\newcommand{\set}[1]{\{#1\}}
\newcommand{\base}{\textbf{Base: }}
\newcommand{\passo}{\textbf{Passo: }}

\begin{document}
\maketitle

\section*{Exercício 41}
Prove que para todo inteiro positivo $n$, vale que $13^n$ pode ser escrito como a soma de dois quadrados.

\begin{proof}
Vamos provar por indução em $n$.

\base Para $n=1$, diretamente $13^1 = 2^2 + 3^2$.

\passo Fixe $n \geq 2$ e suponha para $n-1$, ou seja, nossa hipótese de indução verifica que existem $x,y \in \mathbb{Z}$ tais que $13^{n-1} = x^2 + y^2$.

Queremos provar que existem $z,w \in \mathbb{Z}$ tais que $13^n = z^2 + w^2$.
Observe que $13^n = 13^{n-1} \cdot 13^1$.
Pela base, temos que $13^1 = 2^2 + 3^2$, e pela hipótese de indução, $13^{n-1} = x^2 + y^2$.
Logo,
\begin{align*}
  13^n = 13^{n-1} \cdot 13^1 &= (x^2 + y^2) \cdot (2^2 + 3^2) \\
  &= (2^2 x^2 + 3^2 y^2) + (3^2 x^2 + 2^2 y^2) \\
  &= ((2x)^2 + (3y)^2) + ((3x)^2 + (2y)^2) \; .
\end{align*}
Vamos manipular a expressão para criar 2 quadrados perfeitos.

Observe que, para formar 2 quadrados perfeitos, faltam os termos $\pm \, 2 \cdot 2 \cdot 3 \cdot xy = \pm \, 12xy$.
Logo, ao se somar $+ 12xy$ e $- 12xy$ em $((2x)^2 + (3y)^2) + ((3x)^2 + (2y)^2)$, mantemos a igualdade:
\begin{align*}
  &\big[ ((2x)^2 + (3y)^2) + ((3x)^2 + (2y)^2) \big] + \big[ 12xy - 12xy \big] = (2x + 3y)^2 + (3x - 2y)^2
\end{align*}
Portanto, os $z,w$ que queríamos existem e são $z = (2x + 3y)$ e $w = (3x - 2y)$.

\end{proof}

\section*{Exercício 42}
Uma máquina automática é preenchida com uma quantidade ímpar de exercícios de FUMAC e uma quantidade ímpar de exercícios de Cálculo I.
Sempre que colocamos uma ficha na máquina, ela libera 2 exercícios de uma vez.
A máquina só é abastecida após ficar vazia.
Prove que, antes de ficar vazia, a máquina irá liberar pelo menos um par que é composto por um exercício de FUMAC e um exercício de Cálculo I.

\begin{proof}
Sejam $f,c$ os números não-negativos de exercícios de FUMAC e de Cálculo I, respectivamente.
Como tanto $f$ quanto $c$ são ímpares, então existem $f' , c' \in \mathbb{Z}$ tais que $f = 2f' + 1$ e $c = 2c' + 1$.

Vamos começar a retirada de pares de exercícios das máquinas.

Caso sejam retirados um par FUMAC-Cálculo, temos o que queremos.
Então vamos esgotar nossas possibilidades retirando pares com exercícios iguais.
Observe que é possível retirar, no máximo, $f'$ pares de exercícios de FUMAC e $c'$ pares de exercícios de Cálculo I.
Agora, temos a máquina na seguinte situação:
\begin{itemize}
  \item Tínhamos $f = 2f' + 1$ exercícios de FUMAC. Retiramos $2f'$, então sobrou 1 na máquina.
  \item Tínhamos $c = 2c' + 1$ exercícios de Cálculo I. Retiramos $2c'$, então sobrou 1 na máquina.
\end{itemize}
Como previsto, em última instância sobrou um par FUMAC-Cálculo.

\end{proof}

\section*{Exercício 47}
Prove que para todo inteiro $n \geq 2$, se $x_1,\dots, x_n$ são números reais, então
\[
\sqrt{x_1^2 + \dots + x_n^2} \leq |x_1| + \dots + |x_n| \; .
\]
Obs.: Pode ser útil considerar um número positivo $z$ tal que $z^2 = \sum_{i=1}^{n-1} x_i^2$ .

\begin{proof}
Vamos provar por indução em $n$.

\base Para $n=2$, 
\end{proof}

\end{document}